\documentclass[a4paper,12pt]{article}
\usepackage[utf8x]{inputenc}
\usepackage[english]{babel}
\usepackage{enumerate}
\usepackage{textcomp}
\usepackage{graphicx}
\usepackage{microtype}
\usepackage{hyperref}
\usepackage{url}
\urlstyle{sf}
\usepackage{mathtools}
\usepackage[amssymb]{SIunits} 

\bibliographystyle{utphys}
\newcommand{\code}[1]{\texttt{#1}}
\newcommand{\der}{\mathrm{d}}
\newcommand{\A}{\mathcal{A}}
\newcommand{\N}{\mathcal{N}}

\title{Numerical Code to Calculate Diffractive $\gamma^*$-Nucleus Scattering Cross Section}
\author{Heikki Mäntysaari}
\date{}

\begin{document} 
\maketitle
\begin{center}\today\end{center}
\begin{abstract}
This program calculates numerically both elastic and quasi-elastic diffractive deep inealastic $\gamma^*$-nuclues cross section. This document describes the physical background of the numerical code.
\end{abstract}


\section{Introduction}
For a $\gamma^*A \rightarrow VA$ scattering ($V$ is e.g. a vector meson) we have the following formula for an amplitude \cite{Caldwell:2009ke}
\begin{equation}
	\A^{\gamma^* A}(x,Q,\Delta) =	\sum_f \int \der^2 r \int_0^1 \frac{\der z}{4\pi} \Psi_V^*(r,z,Q) 2\A_{q\bar q}(x,r,\Delta) \Psi(r,z,Q).
\end{equation}
Here $\A_{q\bar q}$ is the scattering amplitude for dipole-nucleus scattering. We also define amplitude and $\der \sigma/\der^2 b$ so that
\begin{equation}
	\frac{\der \sigma}{\der^2 b} = 2\A(x,r,b).
\end{equation}

When we calculate the quasi-elastic cross section we have to sum over all possible excited states of nucleus and any breakup of the nucleus into colorless nuclei. In addition, the cross section must be averaged over nucleon configurations. This is possible to do using the completness relation $\sum_n \Psi_{A_n}(b_1,\dots,b_i) \Psi_{A_n}^*(b_1',\dots,b_i') = \prod_i \delta(b_i-b_i')$. Here $\Psi_{A_n}$ is the wave function of nucleus at state $A_n$. As a result we get
\begin{equation}
\begin{split}
	\label{eq:Asqr}
	|\A^{\gamma^* A}|^2 &= \int \der^2 r \int \frac{\der z}{4\pi} (\Psi_V^*\Psi)(r,z,Q) \int \der^2 r' \int \frac{\der z'}{4\pi} (\Psi_V^*\Psi)(r',z',Q) \\
	&\quad  \times \langle|2\A_{q\bar q}|^2(x,r,r',\Delta)\rangle_N.
\end{split}
\end{equation}
The program calculates the integrals in the same order as how they are written in eq. \eqref{eq:Asqr}. $\langle|\A_{q\bar q}(\Delta)|^2\rangle_N$ is the dipole-nucleus scattering amplitude squared averaged over nucleon configurations and summed over possible final states of nucleus (completness relation is used here when deriving eq. \eqref{eq:Asqr}). $\A(\Delta)$ can be obtained by means of Fourier transformation:
\begin{equation}
	\A_{q\bar q}(\Delta) = \int \der^2 b e^{-ib\cdot \Delta} 2\A_{q\bar q}(b) = i\int \der^2 b e^{-ib\cdot \Delta} \frac{\der \sigma_{q\bar q}}{\der^2 b}
\end{equation}
Notice that
\begin{equation}
	\frac{\der \sigma_{q\bar q}}{\der^2 b} = 2(1-\hat S) = 2\A_{q\bar q}(b),
\end{equation}
where $\hat S$ is the scattering matrix elment. 

In this program $|\A_{q\bar q}(\Delta)|^2$ depends on the given dipole model and it is implemented as \code{Dipxs::DipoleAmplitude\_sqr\_avg}, notice that it returns $|\A(\Delta)|/2^2$. The rest of eq. \eqref{eq:Asqr} is implemented in \code{Calculator} class and does not depend on the used dipole model.

The amplitude squared can be used to obtain the differential cross section:
\begin{equation}
	\frac{\der \sigma}{\der t} = \frac{1}{16\pi} |\A(\Delta)|^2 .
\end{equation}

\section{Dipole cross section}
Code supports a few different models for dipole cross section, and it is quite easy to implement a new one. Just derive a class from class \code{Dipxs}.

\subsection{Ip Non-Sat}
Code: This model is implemented as a class \code{Dipxs\_IPNonSat}.

Ip Non-Sat is the most simple model for dipole-proton cross section. It is derived in e.g. \cite{Caldwell:2009ke} by Kowalski and Caldwell. In this model the dipole-proton amplitude is given by
\begin{equation}
	\label{eq:nonsat-d2b}
	\frac{\der \sigma_{q\bar q}}{\der^2 b} = \frac{\pi^2}{N_C}r^2 \alpha_s(µ^2)xg(x,µ^2) T_p(b). 
\end{equation}
We assume that $N_C=3$. The proton shape function is approximated to be gaussian normalized to unity \cite{Caldwell:2009ke}:
\begin{equation}
	T_p'(b) = \frac{e^{-b^2/(2B_p)}}{2\pi B_p},
\end{equation}
where $B_G=\unit{4.0}{\giga\electronvolt^{-2}}$.

From now on we use the wollowing notation:
\begin{align}
	T_p(b) &= e^{-b^2/(2B_p)} \\
	F(x,r^2) &= \frac{1}{2\pi B_p} \frac{\pi^2}{2 N_C} \alpha_s\left(\mu^2 \right) xg\left(x, \mu^2\right) \, .
\end{align}
Here $\mu^2 = \mu_0^2 + C/r^2$, and we use parameters $\mu_0 = \unit{1.16999996}{\giga\electronvolt^2}$ and $C=4.0$. $xg(x,\mu^2)$ is the dimensionless gluon density, see some notes about it in chapter \ref{gdist}. The two-body correlations are neglected.

This amplitude can be used to (see \cite{Caldwell:2009ke}) derive the amplitude squared for the quasi-elastic dipole-nucleus scattering:
\begin{equation}
\begin{split}
	\langle|\A_{q\bar q}|^2(x,r,r',\Delta)\rangle_N &= 2^2(2\pi B_p)^2 r^2F(x,r)r'^2F(x,r') e^{-B_p \Delta^2} A  \\
	& \quad \times \left[ 1 + (A-1) \left| \int \der^2 b e^{-ib \cdot \Delta} T_A(b)\right|^2 \right] . 
\end{split}
\end{equation}
%Here 
%\begin{equation}
%	\label{eq:kappa}
%	\kappa(r) = \frac{\pi^2}{3}r^2\alpha_s(\mu^2)xg(x,\mu^2).
%\end{equation}

For dipole-proton scattering we just substitute \eqref{eq:nonsat-d2b} into eq. \eqref{eq:Asqr}. We can also derive result for coherent scattering \cite{Caldwell:2009ke}:
\begin{equation}
	|\langle \A_{q\bar q}(x,r,\Delta)\rangle_N|^2 = A^2 2^2 (2\pi B_p)^2 (r^2F(x,r))^2 \left| \int \der^2 b T_A(b) e^{-ib \cdot \Delta} \right|^2 .
\end{equation}

\subsection{IP Sat}
\label{ipsat}
Code: This model is implemented as a class \code{Dipxs\_IPSat}.

IP Sat model takes into account the unitarity requirement of the scattering matrix $S$. In this model the dipole-proton scattering amplitude can be written as \cite{PhysRevD.68.114005}
\begin{equation}
	\label{eq:ipsat-d2b}
	\frac{\der \sigma_{q\bar q}}{\der^2 b} = 2\left[1- \exp \left(-\frac{\pi^2}{N_C\cdot 2}r^2 \alpha_s(µ^2)xg(x,µ^2) \frac{T_p(b)}{2\pi B_p} \right) \right].
\end{equation}
In order to simplify calculation we assume that we can factorize the impact parameter dependency of eq. \eqref{eq:ipsat-d2b} to get
\begin{equation}
	\label{eq:ipsat-d2b-f}
	\frac{\der \sigma_{q\bar q}}{\der^2 b} \approx 2T(b) \left[1 - e^{-r^2F(x,r)} \right].
\end{equation}

This can be generalized to the dipole-nucleus scattering by means of the product of $\hat S$ matrix elements. For dipole-nucleus scattering we use the equation $\hat S(b) = \prod_i \hat S(b-b_i)$. Thus
\begin{equation}
	\label{eq:ipsat-a}
	\A(x,r,b) = 1 - \prod_i \left( 1-\N(r)T(b-b_i) \right),
\end{equation}
and
\begin{equation}
	\frac{\der \sigma_{q\bar q}^A}{\der^2 b} = 2\A(x,r,b).
\end{equation}
Here
\begin{equation}
	\N(r) = 1-e^{-r^2F(x,r)}.
\end{equation}

Neglecting terms that do not contribute at large $\Delta$ and taking the large $A$ limit we obtain the result
\begin{equation}
\label{eq:ipsat-asqr}
\begin{split}
	\langle |\A|^2(x,r,r',\Delta) \rangle_N &= 16\pi B_p \int \der^2 b_2 \sum_{n=1}^A \binom{A}{n} e^{-B_p \Delta^2/n} e^{-2A\pi B_p T_A(b_2)[\N(r)+\N(r')]} \\
	&\quad \times \left(\frac{\pi B_p \N(r)\N(r') 2 T_A(b_2)}{1-2\pi B_p T_A(b_2)[\N(r)+\N(r')]}\right)^n.
\end{split}
\end{equation}

Equation \eqref{eq:ipsat-asqr} can be derived as follows:
\begin{align*}
	\frac{1}{4}&\left \langle \left| \frac{\der \sigma_{q\bar q}^A}{\der^2 b} \right|^2 \right \rangle_N = \int \left(\prod_i \der^2 b_i T_A(b_i) \right) \A(x,r,b) \A^*(x,r',b') \\
	& = 1 - \prod_i \left(1 - \N(r) \int \der^2 b_i T_A(b_i) T(b-b_i) \right) \\
	& - \prod_i \left(1- \N(r') \int \der^2 b_i T_A(b_i) T(b'-b_i)) \right) \\
	& + \prod_i \left[ 1 - \N(r) \int \der^2 b_i T_A(b_i) T(b-b_i) - \N(r')\int \der^2 b_i T_A(b_i)T(b'-b_i)  \right. \\
	&\left. \,\, + \N(r)\N(r')\int \der^2 b_i T_A(b_i) T(b-b_i)T(b'-b_i) \right]
\end{align*}
Now the integrals can be calculated when assuming that $T_A$ is a smooth function when compared with $T$. So in every integral except the last one we change variable to $b_i'=b_i-b$, so we have to integrate $\int \der^2 b_i' T_A(b_i'+b) T(b_i')$. Now $T(b_i')=0$ when $b_i' \sim \unit{1}{\femto\meter}$, and $T_A(b+b_i')\approx T_A(b)$ when $b_i < \unit{1}{\femto\meter}$. Thus 
\begin{equation}
	\int \der^2 b_i T_A(b_i)T(b-b_i) \approx T_A(b) \int \der^2 b_i' T(b_i') = T_A(b)2\pi B_p.
\end{equation}

The last integral can be calculated when noticing that
\begin{equation}
	T(b-b_i)T(b'-b_i) = \exp\left( - \frac{2(b_i - \frac{1}{2}(b+b'))^2 + \frac{1}{2}(b-b')^2}{2B_p} \right).
\end{equation}
Changing integration variable to $b_i' = \frac{1}{2}(b+b')-b_i$ and using the smoothness of $T_A$ we find that
\begin{equation}
	\int \der^2 b_i T_A(b_i)T(b-b_i)T(b'-b_i) \approx T_A\left( \frac{1}{2}(b+b') \right) e^{-(b-b')^2/(4B_p)} \pi B_p.
\end{equation}
Thus we get
\begin{equation}
\begin{split}
	\frac{1}{4}\left \langle \left| \frac{\der \sigma_{q\bar q}^A}{\der^2 b} \right|^2 \right \rangle_N &= 1 - \left(1-\N(r)2\pi B_p T_A(b)\right)^A - \left(1-\N(r')2\pi B_p T_A(b') \right)^A \\
	&\quad +  \bigg[ 1 - 2\pi B_p(\N(r)T_A(b)+\N(r')T_A(b'))  \\
	&\quad \left. + \N(r)\N(r')\pi B_p T_A\left(\frac{1}{2}(b+b')\right) e^{-(b-b')^2/(4B_p)} \right]^A.
\end{split}
\end{equation}

In order to calculate the amplitude in momentum space we have to compute the fourier transformation:
\begin{equation}
	\langle |\A(x,r,r',\Delta)|^2 \rangle = \int \der^2 b \der^2 b' e^{-i(b-b')\cdot \Delta} \left \langle \left| \frac{\der \sigma_{q\bar q}^A}{\der^2 b} \right|^2 \right \rangle_N.
\end{equation}
We can compute this by making a change of variables to $b_1 = b-b'$ and $b_2 = \frac{1}{2}(b+b')$. As the function is highly oscillatory, we can neglect terms not containing $e^{-b_1^2}$ due to the strong suppression. So we neglect everything but the last $[\dots ]^A$ term which we expand using the binome factor. We also use the smoothness of $T_A$: now $b_1$ is small so $T_A(b)=T_A(\frac{1}{2}(b_1+2b_2)) \approx T_A(b_2)$ and similarly $T_A(b')\approx T_A(b_2)$. So at the end we get
\begin{equation}
\begin{split}
	\langle |&\A(x,r,r',\Delta)|^2 \rangle = 4\int \der^2 b_1 \der^2 b_2 e^{-ib_1 \cdot \Delta} \sum_{n=1}^A \binom{A}{n} e^{-nb_1^2/(4B_p)}  \\
	& \times \left(1-2\pi B_p T(b_2)[\N(r)+\N(r')] \right)^{A-n} \left( T_A(b_2)\pi B_p \N(r)\N(r') \right)^n 
\end{split}
\end{equation}
Calculating $b_1$ integral and writing $()^{A-n} = ()^A ()^{-n}$, and taking the limit of large $A$ we can write
\begin{equation}
\begin{split}
	\langle |&\A(x,r,r',\Delta)|^2 \rangle = 16\pi B_p \int \der^2 b_2  \sum_{n=1}^A \binom{A}{n} \frac{1}{n} e^{-B_p\Delta^2/n}  \\
	&\times e^{-A\cdot 2\pi B_p T_A(b_2)[\N(r)+\N(r')]} \left( \frac{\pi B_p \N(r)\N(r')T_A(b_2)}{1-2\pi B_p T_A(b_2) [ \N(r)+\N(r')]} \right)^n.
\end{split}
\end{equation}
Numerical calculations show that terms with $n>1$ can be neglected.

For coherent scattering we can derive \cite{PhysRevD.68.114005} the formula which is valid in smooth nucleus limit:
\begin{equation}
	\frac{1}{2}\left\langle \A(x,r,\Delta)\right\rangle_N\ = 1 - \exp\left( 1 - A\frac{1}{2}T_A(b)\sigma_\text{dip}^p(r,x) \right).
\end{equation}
Here
\begin{equation}
	\sigma_\text{dip}^p(r,x) = \int \der^2 b \frac{\der \sigma_{q\bar q}^p}{\der^2 b} = 2\cdot 2\pi B_p[1-e^{-r^2F(x,r)}]
\end{equation}
is the total $q\bar qp$ cross section.

It is also interesting to see what happens if $\N(r)=r^2F(x,r)$. This means that we use the result \eqref{eq:nonsat-d2b} for dipole-proton scattering (so, we do not require the unitarity for dipole-proton scattering), but we still take into account the possibility to scatter from multiple nucleons at the same time. This possibility is implemented in \code{Dipxs\_IPSat} class, and can be used if the mode is set to \code{IPSAT\_MODE\_NONSAT\_P}.

For dipole-proton scattering we make the same approximation as it is done when deriving eq. \eqref{eq:ipsat-asqr}:
\begin{equation}
	\frac{\der \sigma_{q\bar q}}{\der^2 b} = 2 T_p(b)\left( 1-e^{-r^2F(x,r) } \right).
\end{equation}

\subsection{IIM}
The heavy-quark improved Iancu-Itakura-Munier (IIM) model is defined by parts so that when $rQ_s(x)>1$, the dipole-proton scattering amplitude is about $1$. When $rQ_s(x) \ll 1$, the amplitude is also small. Here $Q_s(x)$ is the saturation scale parametrised by
\begin{equation}
	Q_s(x) = Q_0 \frac{x_0}{x}^{\lambda/2}.
\end{equation}
We take the parameters from \cite{Marquet:2007nf}: $Q_0 = \unit{1.0}{\giga\electronvolt}$, $\lambda = 0.2197$ and $x_0 = 1.632 \cdot 10^{-5}$.

In IIM model the dipole-proton amplitude is factorized as
\begin{equation}
	\label{eq:iim-a}
	\A_{q\bar q} = S(b) N(rQ_s,x).
\end{equation}
Here the impact parameter dependence is factorized in $S(b)$ and it is assumed to be gaussian
\begin{equation}
	S(b) = e^{-b^2/(2B_D)}.
\end{equation}
The parameter $B_D = \unit{5.591}{\giga\electronvolt^{-2}}$.

Moreover, $N$ can be written as
\begin{equation}
N(rQ_s,x) = \begin{cases}
	N_0\left( \frac{rQ_s}{2} \right)^{2\gamma_c} \exp \left[ -\frac{2 \ln^2 (rQ_s/2) }{\eta \lambda \ln (1/x)} \right] & \text{for } rQ_s \leq 2  \\
	1-e^{-4\alpha \ln^2 (\beta rQ_s)} & \text{for } rQ_s > 2	
	\end{cases}
\end{equation}
In this heavy-quark improved model the parameters are: $\eta = 9.9$, $\gamma_c = 0.7376$, $\lambda=0.2197$ and $N_0 = 0.7$. $\alpha$ and $\beta$ are determined from the requirement that $N$ and its derivative are continuous at $rQ_s=2$. Using the parameters introduced here one finds that $\alpha = 0.615065$ and $\beta = 1.00642$.

%This model can be generalized to dipole-nucleus scattering in two different ways. If we neglect the possibility to scatter from multiple nucleons at the same time we can write $S(b) = \sum_i S(b-b_i)$ and get the result
%\begin{equation}
%\begin{split}
%	|\A_{q\bar q}|^2(x,r,r',\Delta) &= 4\pi^2B_D^2 e^{-B_D \Delta^2} N(rQ_s,x)N(r'Q_s,x) A \\
%	&\quad \times \left[1 + (A-1) \left| \int \der^2 b e^{-ib \cdot \Delta} T_A(b) \right|^2 \right] . 
%\end{split}
%\end{equation}

%Better way is to calculate
This mode lcan be generalized to dipole-nucleus scattering similarly as we did with the IPsat model: we write the amplitude as a product of $\hat S$ matrix elements (note here that $\hat S$ refers to the scattering matrix and $S(b)$ is the impact-parameter dependence of the dipole-proton amplitude). 
\begin{equation}
	\hat S = 1-S(b)N(rQ_s,x).
\end{equation}

Writing $\hat S^A(b) = \prod_i \hat S(b-b_i)$ we figure out that
\begin{equation}
	\A(x,r,b) = 1 - \prod_i \left( 1-S(b)N(rQ_s,x) \right).
\end{equation}
Comparing this with eq. \eqref{eq:ipsat-a} one can see the similar form of the amplitude. Thus we can proceed as with the IPSat model and finally obtain the formula similar to \eqref{eq:ipsat-asqr} (large $\Delta$ and $A$). Only difference is that now $\N(r)=N(rQ_s,x)$.

\section{Vector Meson Wave Function}
We have to calculate the overlap between a virtual photon and a vector meson wave functions. It is quite easy to implement a new model, just derive the class from the class \code{WaveFunction}.

\subsection{Gaus-LC}
In case of vector meson production one can use the Gaus-LC model to get the overlap between $\gamma^*$ and the VM \cite{Kowalski:2006hc}.

In this model we get the overlaps for transversial and longitudinal $\gamma^*$ polarisations as
\begin{align}
	|\Psi |^2_T &= e_f \frac{N_c}{\pi z(1-z)} \left[ m_f^2 K_0(\epsilon r)\phi_T(r,z) - (z^2 + (1-z)^2) \epsilon K_1(\epsilon r) \partial_r \phi_T(r,z) \right] \\
	|\Psi |^2_L &= e_f \frac{N_c}{\pi} 2Q z (1-z)K_0(\epsilon r) \left[ M_V \phi_L(r,z) + \delta \frac{m_f^2 - \nabla_r^2}{M_V z(1-z)}\phi_L(r,z) \right].
\end{align}
Here
\begin{align}
	\phi_T(r,z) &= N_T \left[ z(1-z)\right]^2 e^{-r^2/(2R_T^2)}, \\
	\phi_L(r,z) &= N_L z(1-z) e^{-r^2/(2R_L^2)}.
\end{align}
We also use notation $\nabla_r^2 = (1/r)\partial_r + \partial_r^2$. $\epsilon^2=z(1-z)Q^2+m_f^2$.

Parameters for $J/\Psi$ production are: $M_V = \unit{3.097}{\giga\electronvolt}$, $m_f = \unit{1.4}{\giga\electronvolt}$, $N_T = 1.23$, $R_T^2 = \unit{6.5}{\giga\electronvolt^{-2}}$, $N_L=0.83$, $R_L^2 = \unit{3.0}{\giga\electronvolt^{-2}}$, $e_f = 2/3 e$, $\delta = 0$ or 1.


\section{Gluon Distribution}
\label{gdist}
Code: \code{GDist} class. See also \code{src/gdist/gdist\_dglap.h}.

\bibliography{refs}

\end{document}


